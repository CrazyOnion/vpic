\documentclass[twocolumn,10pt]{article}

\usepackage{amsmath}
\usepackage{amsfonts}
\usepackage{amssymb}
\usepackage{geometry}

\geometry{left=2cm,right=2cm,top=3cm,bottom=2.5cm}

\renewcommand{\vec}[1]{\mathbf{#1}}
\newcommand{\omegap}{\omega_{\mathrm{pe}}}
\newcommand{\ncrit}{n_{\mathrm{cr}}}
\newcommand{\vth}{v_{\mathrm{th}}}
\newcommand{\ldebye}{\lambda_{\mathrm{D}}}

\title{Units in VPIC}
\author{A. Seaton}
\begin{document}
	\maketitle

	Internally, the code uses the SI form for Maxwell's equations and the Lorentz force (i.e. not the Guassian form). However, the code requires the user to specify various constants, and these determine the unit system. In particular, the user defines:

	\begin{itemize}
		\item $l_{\mathrm{deck}}$: an implicit unit of length for the spatial grid
		\item $c_{\mathrm{deck}}$: the speed of light in vacuum
		\item $\varepsilon_{0,\mathrm{deck}}$: the vacuum permittivity
		\item $r_{\alpha,\mathrm{deck}} \equiv (q_{\alpha}/m_{\alpha})_{\mathrm{deck}}$: The charge/mass ratio of species $\alpha$.
	\end{itemize}

	\noindent Together these fully determine the unit system used by the code. One reason for this flexibility (versus having a fixed unit system) is that VPIC uses single-precision floating point numbers. This means that all values, including at intermediate steps in calculations, must have absolute value $|x|$ between $\sim 10^{-38}$ and $\sim 10^{38}$. Depending on the problem, a unit system such as SI may require values smaller or larger than these.

	\section{Unit Conversions}

	Given an input deck and simulation data, we want to convert the output to a familiar unit system. The quantities we are typically interested in are length, time, and the electromagnetic fields. Taking our base units as length, time, mass, and charge, we define normalisation factors $L$, $T$, $M$, and $Q$ so that a code quantity $f_{c}$ is converted to a value $f$ in our preferred unit system using $f = Ff_c$. To determine what our base unit normalisation factors are, we write out the constraints specified in the deck:

	\begin{align*}
		l_{\mathrm{deck}} &= L, &
		c_{\mathrm{deck}} &= c\frac{T}{L}, \\
		\varepsilon_{0,\mathrm{deck}} &= \varepsilon_0\frac{ML^3}{Q^2T^2}, &
		r_{\alpha,\mathrm{deck}} &= r_{\alpha}\frac{M}{Q},
	\end{align*}

	\subsection{General Conversion Factors}

	Solving the above system for $L$, $T$, $M$, and $Q$, we find:

	\begin{align}
		L &= l_{\mathrm{deck}}, \\
		T &= l_{\mathrm{deck}}\frac{c_{\mathrm{deck}}}{c}, \\
		M &= l_{\mathrm{deck}}\left(\frac{c}{c_{\mathrm{deck}}}\right)^2\frac{\varepsilon_0}{\varepsilon_{0,\mathrm{deck}}}\left(\frac{r_{\alpha,\mathrm{deck}}}{r_{\alpha}}\right)^2, \\
		Q &= l_{\mathrm{deck}}\left(\frac{c}{c_{\mathrm{deck}}}\right)^2\frac{\varepsilon_0}{\varepsilon_{0,\mathrm{deck}}}\frac{r_{\alpha,\mathrm{deck}}}{r_{\alpha}}.
	\end{align}

	\noindent The electric and magnetic field conversion factors $E$ and $B$ are then given by

	\begin{align}
		E &= \frac{ML}{QT^2} = \frac{1}{l_{\mathrm{deck}}}\frac{r_{\alpha,\mathrm{deck}}}{r_{\alpha}}\left(\frac{c}{c_{\mathrm{deck}}}\right)^2 \\
		B &= E\frac{T}{L} = \frac{1}{l_{\mathrm{deck}}}\frac{r_{\alpha,\mathrm{deck}}}{r_{\alpha}}\frac{c}{c_{\mathrm{deck}}}
	\end{align}

	\subsection{Particle Momenta}

	Particle momenta are the only user-facing exception to the unit system defined above. The momenta are normalised to produce a dimensionless momentum

	\begin{equation}
		u \equiv \frac{p}{mc}.
	\end{equation}

	This dimensionless momentum is used for example when loading particles, however the momentum density given in the hydro dumps is in the unit system defined above.

	\subsection{Common Unit Systems}

	Various common choices are made for the unit system, and we discuss a few of them here. In general, aside from ensuring that values do not exceed the limits for single-precision floating point numbers, it is often desireable to choose the unit system to ease interpretation of simulation results.

	\subsubsection{Electrostatic Plasma Units}

	In an electrostatic problem simulating an electron plasma the natural length and time-scales are the Debye length $\ldebye$ and inverse plasma frequency $\omegap^{-1}$. These are also the typical scales for the grid and time-step. Then choosing the electron charge/mass ratio so that $e/m_e$ becomes unity reduces the equation of motion of an electron to

	\begin{equation*}
		\ddot{\vec{r}} = \vec{E}.
	\end{equation*}

	\noindent Finally setting $\varepsilon_0=1$ means that Gauss' law is also simplified

	\begin{equation*}
		\nabla \cdot \vec{E} = \rho.
	\end{equation*}

	The deck constants are therefore

	\begin{align*}
		l_{\mathrm{deck}} &= \ldebye, &
		c_{\mathrm{deck}} &= \frac{c}{\vth}, \\
		\varepsilon_{0,\mathrm{deck}} &= 1, &
		r_{e,\mathrm{deck}} &= -1,
	\end{align*}

	\noindent giving the unit conversions as

	\begin{align*}
		L &= \ldebye, &
		T &= \omegap^{-1}, &
		M &= N_{\mathrm{D}}m_e, \\
		Q &= N_{\mathrm{D}}e, &
		E &= \frac{m_e}{e}\vth\omegap, &
		B &= \frac{m_e}{e}\omegap,
	\end{align*}

	\noindent where $N_{\mathrm{D}} \equiv n_e\ldebye^3$ is the number of electrons in a `Debye cube'.

	This unit system allows for some convenient sanity checks. As mentioned above, the cell size and time-step must be around $1$. Additionally, if the electron mass in these units is not significantly smaller than 1, then the plasma is not ideal and simulating it with VPIC is not appropriate. Interpretation of simulation results may also be more convenient. For example the normalised electric field $E_c$ is the electron quiver velocity in a plasma wave divided by the thermal velocity - an important parameter for determining whether wavebreaking will occur. Similarly the wavenumber of a plasma wave is normalised such that $k_c = k\ldebye$, which determines the Landau damping rate.

	\subsubsection{Electromagnetic Plasma Units}

	For an electromagnetic problem, the electrostatic units above might be modified so that the length-scale becomes the electron skin depth $c/\omegap$. The Lorentz force and Maxwell's equations become

	\begin{equation*}
		\ddot{\vec{r}} = \vec{E} + \vec{v} \times \vec{B},
	\end{equation*}
	\noindent and
	\begin{align*}
		\nabla \times \vec{E} &= -\partial_t \vec{B}, \\
		\nabla \times \vec{B} &= \left(\vec{J} + \partial_t \vec{E}\right).
	\end{align*}

	\noindent The deck constants are then

	\begin{align*}
		l_{\mathrm{deck}} &= \frac{c}{\omegap}, &
		c_{\mathrm{deck}} &= 1, \\
		\varepsilon_{0,\mathrm{deck}} &= 1, &
		r_{e,\mathrm{deck}} &= -1,
	\end{align*}

	\noindent so that the conversion factors are now

	\begin{align*}
		L &= \frac{c}{\omega_{\mathrm{pe}}}, &
		T &= \frac{1}{\omega_{\mathrm{pe}}}, \\
		M &= n_e\frac{c^3}{\omegap^3}m_e, &
		Q &= n_e\frac{c^3}{\omegap^3}e, \\
		E &= \frac{m_e}{e}\omegap c, &
		B &= \frac{m_e}{e}\omegap.
	\end{align*}

	\noindent Note that there is no longer any dependence on temperature.

	\subsubsection{Laser-Plasma Units}

	For laser-plasma interaction problems, the unit system can again be modified by replacing the time-scale with the inverse laser frequency, so that unit system no longer depends on the electron density used to calculate $\omega_{\mathrm{pe}}$. The deck constants are then

	\begin{align*}
		l_{\mathrm{deck}} &= \frac{1}{k_0}, &
		c_{\mathrm{deck}} &= 1, \\
		\varepsilon_{0,\mathrm{deck}} &= 1, &
		r_{e,\mathrm{deck}} &= -1,
	\end{align*}

	\noindent where $k_0 \equiv \omega_0/c$ is the vacuum laser wavenumber. The conversion factors then become

	\begin{align*}
		L &= \frac{1}{k_0}, &
		T &= \frac{1}{\omega_0}, \\
		M &= n_ek_0^3m_e, &
		Q &= n_ek_0^3e, \\
		E &= \frac{m_e}{e}\omega_0 c, &
		B &= \frac{m_e}{e}\omega_0.
	\end{align*}

	In this unit system the simulation cell size and time step should satisfy $\Delta x \ll 1$ and $\Delta t \ll 1$ in order for the field solver to accurately model the laser. It is also necessary to verify that the cell size resolves the Debye length, which is often a more restrictive condition. Additionally, the normalised electric (or magnetic) field is the parameter $a_0$. For $a_0 \ll 1$ this is the electron quiver velocity divided by the speed of light, while $a_0 \gtrsim 1$ indicates a relativistic field intensity which will lead to effects such as relativistically induced transparency and relativistic self-focusing.

	\section{Species \& Particle Weighting}

	To define a particle species in VPIC, the user specifies the species mass $m_{\mathrm{s}}$ and charge $q_{\mathrm{s}}$. Additionally, each particle is individually assigned a weighting factor $w$. These must be consistent with each other and the above unit system to produce the desired behaviour.

	\subsection{Choosing Charge Mass Ratio}

	The motion of a physical particle with charge $q_{\mathrm{p}}$ and mass $m_{\mathrm{p}}$ in the EM fields only has explicit dependence on the charge-mass ratio $r \equiv q_{\mathrm{p}}/m_{\mathrm{p}}$. This is also the case for macroparticles in VPIC's particle push. Therefore the corresponding macroparticle species must be defined such that its charge $q_{\mathrm{s}}$ and mass $m_{\mathrm{s}}$ satisfy $q_{\mathrm{s}}/m_{\mathrm{s}}=r$.

	\subsection{Choosing Macroparticle Weight}
	When calculating the fields generated by a macroparticle, the code uses the species charge and macroparticle weight to calculate a total charge for the macroparticle:

	\begin{equation}
		q_{\mathrm{MP}} = wq_{\mathrm{s}}.
	\end{equation}

	\noindent For this to result in an amount of current deposition consistent with the desired physical particle density, the macroparticle weight should be defined such that the charge contained in volume $V$ is the same for the macroparticles as for physical particles, i.e.

	\begin{align*}
		q_{\mathrm{p}}n_{\mathrm{p}}V &= q_{\mathrm{MP}}n_{\mathrm{MP}}V \\
			&= wq_{\mathrm{s}}n_{\mathrm{MP}}V,
	\end{align*}

	\noindent where $n_{\mathrm{p}}$ and $n_{\mathrm{MP}}$ are the physical particle and macroparticle number densities (macroparticles per unit volume) respectively. So, the particle weight must satisfy

	\begin{align}
		w = \frac{q_{\mathrm{p}}n_{\mathrm{p}}}{q_{\mathrm{s}}n_{\mathrm{MP}}} = \frac{\rho_{\mathrm{p}}}{\rho_{\mathrm{s}}}.
	\end{align}

	\noindent This illustrates that there is a degree of freedom available in that the species charge defined in the deck does not have to be the same as the physical species charge.

	\subsection{Consequences of $q_{\mathrm{s}} \neq q_{\mathrm{p}}$}

	Choosing $q_{\mathrm{s}} \neq q_{\mathrm{p}}$ means that the particle weight can no longer be described intuitively as the ``number of physical particles per macro-particle''. This means that without additional information, the code has no way of calculating numbers of physical particles. This can be seen by rearranging the above equation:

	\begin{align}
		n_{\mathrm{p}} = wn_{\mathrm{MP}}\frac{q_{\mathrm{s}}}{q_{\mathrm{p}}}.
	\end{align}

	\noindent In order to calculate the number of physical particles, the value of the physical particle charge is required. VPIC has been carefully designed to produce valid output despite this, and for example is why the code outputs charge density rather than number density. The number density of phyical particles is often more useful, but the code has no way of calculating it in general.

\end{document}
